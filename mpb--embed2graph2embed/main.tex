% Status as of 04.08.2020 15:00 CET bm
%%%%%%%%%%%%%%%%%%%%%%%%%%%%%%%%%%%%%%%%%%%
\documentclass[review]{elsarticle}

\usepackage{lineno}
\usepackage{easylist}
\usepackage{amssymb}
\usepackage{float}
\usepackage{amsmath}
\usepackage{subcaption}
\usepackage[breaklinks]{hyperref}
\usepackage{url}
\usepackage{textcomp}
\usepackage{verbatim}
\usepackage[ruled,vlined]{algorithm2e}
\usepackage{mathtools}
%\usepackage[ampersand]{easylist}
\setcounter{tocdepth}{3}
\usepackage{graphicx}
\usepackage{pgfplots}
\pgfplotsset{compat=1.14}
\pgfplotsset{compat=newest}
\pgfplotsset{plot coordinates/math parser=false}
\usepackage{tikzscale}
\usetikzlibrary{matrix,chains,positioning,decorations.pathreplacing,arrows}
\usepackage{tikz-qtree,tikz-qtree-compat}
\usetikzlibrary{calc}
\modulolinenumbers[5]
\journal{Journal of cool mini-projects}

%%%%%%%%%%%%%%%%%%%%%%%
%% Elsevier bibliography styles
%%%%%%%%%%%%%%%%%%%%%%%
%% To change the style, put a % in front of the second line of the current style and
%% remove the % from the second line of the style you would like to use.
%%%%%%%%%%%%%%%%%%%%%%%

%% Numbered
%\bibliographystyle{model1-num-names}

%% Numbered without titles
%\bibliographystyle{model1a-num-names}

%% Harvard
%\bibliographystyle{model2-names.bst}\biboptions{authoryear}

%% Vancouver numbered
%\usepackage{numcompress}\bibliographystyle{model3-num-names}

%% Vancouver name/year
%\usepackage{numcompress}\bibliographystyle{model4-names}\biboptions{authoryear}

%% APA style
%\bibliographystyle{model5-names}\biboptions{authoryear}

%% AMA style
%\usepackage{numcompress}\bibliographystyle{model6-num-names}

%% `Elsevier LaTeX' style
\bibliographystyle{elsarticle-num}

%%%%%%%%%%%%%%%%%%%%%%%

\begin{document}
\begin{frontmatter}

\title{Mini-project A: \\ Traditional recommenders vs. (graph) neural embeddings }

\author[TUG,MUG]{Supervisor: Bernd Malle}
\author[TUG,MUG]{\small \\Professor: Andreas Holzinger}
% \cortext[mycorrespondingauthor]{Corresponding author}
% \ead{bernd.malle@medunigraz.at}

\address[TUG]{Graz University of Technology, Austria}
\address[MUG]{Medical University Graz, Austria}


\begin{abstract}

Improve / complement recommendations via graph representations (neural embeddings).

\end{abstract}

\begin{keyword}
similarity, graph representation learning, graph embeddings, recommendations
\end{keyword}

\end{frontmatter}


\section{Motivation}
\label{sect:motivation}

a graph, enabling the application of a wealth of graph theoretical algorithms 

\section{Goals}
\label{sect:goals}

Our overarching goal is to...

\begin{figure}[hbt!]
\begin{center}
% \includegraphics[width=\textwidth]{figures/transductive_learning}
\end{center}
\caption{General overview of a (distributed) graph-based ML pipeline in the medical domain}
\label{fig:Figure-1}
\end{figure}

\section{Section on everything}
\label{section_everything}


\section{Procedure}
\label{sect:procedure}



\bibliography{references}

\end{document}
